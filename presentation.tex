\documentclass[aspectratio=1610, professionalfonts, 13pt]{beamer}

% Lade das TU Dortund Theme von Max Nöthe
\usefonttheme[onlymath]{serif}
\usetheme[showtotalframes]{tudo}

% Lade richtiges Sprachpaket
\ifluatex
    \usepackage{polyglossia}
    \setmainlanguage{english}
\else
    \ifxetex
        \usepackage{polyglossia}
        \setmainlanguage{german}
    \else
        \usepackage[german]{babel}
    \fi
\fi

% Lade wichtige Mathematikpakete
\usepackage{amsmath}
\usepackage{amssymb}
\usepackage{mathtools}
\usepackage{cancel}
\usepackage[
  locale=DE,                   % deutsche Einstellungen
  separate-uncertainty=true,   % Immer Fehler mit \pm
  per-mode=symbol-or-fraction, % m/s im Text, sonst Brüche
]{siunitx}
\usepackage[absolute,overlay]{textpos}
\usepackage{framed}
\usepackage{multicol}
\usepackage{setspace}
\usepackage{graphicx}
\usepackage{booktabs}
\usepackage{caption}
\usepackage{appendixnumberbeamer}
\usepackage{tikz}
\usepackage[export]{adjustbox}
\usepackage{color}


% Lade Paket zur Nutzung von Schleifen
\usepackage{forloop}

% ------------------------- Präsentationsinformationen -------------------------

% Titel:
\title{\textbf{Analysis Of The Crab Nebula Using FACT's Photon Stream Data}}
% Autoren:
\author[K.\ Sedlaczek]{\textit{Kevin Sedlaczek}, Maximilian Nöthe for the FACT-Collaboration}


% Datum:
\date{20. March 2018}
\date[20. March 2018]{DPG-Frühjahrstagung 2018 Würzburg}
% Lehrstuhl/Fakultät:
\institute[TU Dortmund]{Lehrstuhl für Experimentelle Physik 5}

\institute[%
  {\includegraphics[height=\headerheight]{logos/fact.pdf}}%
  \hspace{1em}%
  {\includegraphics[height=\headerheight]{logos/e5logo.pdf}}%
]{
  {\includegraphics[height=0.75cm]{logos/tu.pdf}}%
  \hspace{1em}%
  {\includegraphics[height=0.75cm]{logos/ethz.pdf}}%
  \hspace{1em}%
  {\includegraphics[height=0.75cm]{logos/isdc.pdf}}%
  \hspace{1em}%
  {\includegraphics[height=0.75cm]{logos/uniwue.pdf}}%
}

% Titelgrafik:
% \titlegraphic{\includegraphics[width=0.4\textwidth]{logos/FACTLogo_preliminary.png}\hfill\includegraphics[width=0.4\textwidth]{logos/e5logo_green_text.pdf}}


\begin{document}

\maketitle

\section{The First G-APD Cherenkov Telescope}

\begin{frame}[t]{The First G-APD Cherenkov Telescope}
  \begin{multicols}{2}
    \vspace*{\fill}
    \begin{figure}
        \centering\includegraphics[width=0.5\textwidth]{fig/fact.png}
    \end{figure}
    \columnbreak
    \vspace*{\fill}
      \begin{itemize}
        \item located on Roque des los Muchachos, La Palma
        \item build to demonstrate novel light sensors \textit{silicon photo multipliers} (SiPM)
        \item offer possibility to operate under much brighter light conditions
        \item camera has single photon resolution
    \end{itemize}
    \vspace*{\fill}
  \end{multicols}
\end{frame}

\section{Photon Stream}

\begin{frame}[t]{The Photon Stream Data}
  \begin{multicols}{2}
  \begin{figure}
      \centering
      \includegraphics[width=0.5\textwidth]{fig/standard.png}
      \caption{Standard FACT events.}
  \end{figure}
  \columnbreak
  \textbf{Photon Stream:}
  \begin{itemize}
      \item FACT records data in format close to readout hardware
      \item not intended as physics format \\ $\rightarrow$ \textbf{Photon Stream}
      \item superposition of multiple photon signals
      \item reconstruct arrival time of single photons by substracting their pulse shapes
      \item list of lists of arrival times
  \end{itemize}
  \end{multicols}

\end{frame}

\begin{frame}[t]{The Photon Stream Data}
    \begin{multicols}{2}
    \begin{figure}
        \centering
        \begin{overprint}
            \onslide<1>\includegraphics[width=0.6\textwidth]{fig/event/example_event_small-14.png}
            \caption{Represent \textbf{I}maging \textbf{A}tmospheric \textbf{C}herenkov \textbf{T}elescope events using single photons}
            \onslide<2>\includegraphics[width=0.6\textwidth]{fig/event/example_event_small-15.png}
            \caption{Represent \textbf{I}maging \textbf{A}tmospheric \textbf{C}herenkov \textbf{T}elescope events using single photons}
    \end{overprint}
    \end{figure}
    \columnbreak
    \begin{itemize}
        \item smaller file size: possible to compress all FACT data to fit on one 10TB drive
        \item simplify \textit{exchange} and \textit{analysis}, gain timing knowledge
        \item do cluster based image cleaning
        \item do physics analysis on an SiPM based IACT
    \end{itemize}
    \end{multicols}
\end{frame}



\section{Data Set: FACT open data crab sample}

\begin{frame}[t]{The Data set: FACT open data crab sample}
    \begin{multicols}{2}
        \begin{figure}
            \centering
            \includegraphics[width=0.4\textwidth]{fig/cond.png}
        \end{figure}
    \columnbreak
    \vspace*{\fill}
        \begin{itemize}
            \item \url{https://fact-project.org/data}
            \item Crab Nebula observations from November 2013
            \item including gamma-ray and proton simulations
            \item $17.7$ hours of observations
        \end{itemize}
    \vspace*{\fill}
  \end{multicols}
\end{frame}

\section{Analysis}

\begin{frame}[t]{Analysis}
\begin{description}[Crab Nebula]
    \item[aim] test the Photon Stream data in a physics analysis
    \item[Crab Nebula] well measured source of cosmic gamma rays $\rightarrow$ comparative analysis
\end{description}
FACT Analysis chain:
    \begin{enumerate}
        \item calibration of data
        \item image cleaning $\rightarrow$ pure shower image remains
    \end{enumerate}
\textbf{{\color{tugreen} This Analysis}} so far:
    \begin{enumerate}\addtocounter{enumi}{2}
        \item Parametrization: calculate useful parameters
        \item distinguish protons (background) from gammas
        \item reconstruct direction and energy of particles
    \end{enumerate}
\begin{framed}
    \centering
    This is only the classical analysis approach so far!
\end{framed}
\end{frame}

\begin{frame}[t]{Parameterization}
    \begin{multicols}{2}
        \begin{figure}
            \centering
            \includegraphics[width=0.5\textwidth]{fig/hillas.png}
        \end{figure}
    \columnbreak
Hillas parameters: \\
    \begin{itemize}
        \setlength\itemsep{1em}
        \item \textbf{{\color{tugreen} size}}: number of photons in cluster
        \item \textbf{{\color{tugreen} length}}: std. dev. along long half-axis
        \item \textbf{{\color{tugreen} width}}: std. dev. along short half-axis
        \item \textbf{{\color{tugreen} delta}}: angle between length and disp
        \item \textbf{{\color{tugreen} skewness/ kurtosis}}: higher order statistical moments along half-axes
    \end{itemize}
    \end{multicols}
\end{frame}

\begin{frame}[t]{Parameterization}
    \begin{multicols}{2}
        \begin{figure}
            \centering
            \includegraphics[width=0.5\textwidth]{fig/disp.png}
        \end{figure}
    \columnbreak
Source position reconstruction via disp-method:
    \begin{itemize}
        \item \textbf{{\color{tugreen} |disp|}}: distance from centre of gravity to target
        \item \textbf{{\color{tugreen} sgn(disp)}}: Head/Tail-Disambiguation
        \item \textbf{{\color{tugreen} theta}}: distance between reconstructed and true origin
        \item \textbf{{\color{tugreen} skewness/ kurtosis}}: higher order statistical moments along half-axes in cluster system
    \end{itemize}
    \end{multicols}
\end{frame}

\begin{frame}[t]{Tools}
Machine learning with FACT classifier-tools, using 5-fold cross validation  \url{https://github.com/fact-project/classifier-tools} \\
\textbf{{\color{tugreen} Energy estimation}}:
\begin{itemize}
    \item random forest regressor (30 estimators): \texttt{klaas\_energy\_regressor}
\end{itemize}
\textbf{{\color{tugreen} Gamma-hadron-separation}}:
\begin{itemize}
    \item random forest classifier (50 estimators): \texttt{klaas\_separation\_model}
\end{itemize}
\textbf{{\color{tugreen} Origin reconstruction}}:
\begin{itemize}
    \item two step task: regression of |disp| and classification of sgn(disp)
    \item random forest regressor and classifier: \texttt{klaas\_disp\_regressor}
\end{itemize}
\end{frame}

\section{Results}
\begin{frame}[t]{Energy estimation}

Mean $\text{R}^2$ score from CV: $0.6065\,\pm\,0.0092$

\begin{figure}
    \centering
    \includegraphics[width=0.6\textwidth,page=1]{fig/energy-performance.pdf}
\end{figure}
\end{frame}

\begin{frame}[t]{Separation}

Trained on $100\,000$ signal and background events
\begin{figure}
    \centering
    \includegraphics[width=0.6\textwidth,page=1]{fig/separation_performance.pdf}
\end{figure}
\end{frame}

\begin{frame}[t]{Origin reconstruction}
\begin{itemize}
    \item mean accuracy for sgn(disp): $\SI{75.99(22)}{\percent}$
    \item mean $\text{R}^2$ score for disp: $0.5429\,\pm\,0.0036$
\end{itemize}
\begin{multicols}{2}
\begin{figure}
    \centering
    \includegraphics[width=0.48\textwidth]{fig/theta_whole.pdf}
\end{figure}
\columnbreak
\begin{figure}
    \centering
    \includegraphics[width=0.48\textwidth]{fig/theta_whole.pdf}
\end{figure}
\end{multicols}
\end{frame}


\section{Summary}



\begin{frame}[t]{Summary}
Works well already! \\
\begin{table}
  \centering
  \begin{tabular}{c
                  c
                  S[table-format=1.2]
                  S[table-format=1.4(2)]
                  S[table-format=1.4(2)]}
    \toprule
    {} & {significance crab} & {separation AUC ROC} & {sign accuracy} & {disp $\text{R}^2$} \\
    \midrule
    This analysis & $8.3\,\sigma$ & 0.83 & 0.7599(22)  & 0.5429(36)  \\
    Std analysis & $24.2\,\sigma$ & 0.89 & 0.75        & 0.6(2)  \\
    \bottomrule
  \end{tabular}
\end{table}
% \vspace{10px}
% \textbf{{\color{tugreen} Comparison with FACT standard analysis}}
% \begin{itemize}
%     \item significance on crab sample detection: $24.2\,\sigma$ as compared to $X\,\sigma$
%     \item separation AUC ROC: $0.89$ as compared to $0.83$ (this analysis)
%     \item disp method:
% \end{itemize}
\vspace{20px}
\textbf{{\color{tugreen} Outlook}}
\begin{itemize}
    \item run analysis on more data
    \item apply cuts on data
    \item improve hyper-parameters of classifier-tools and clustering
    \item improve feature selection $\rightarrow$ feature engineering
\end{itemize}
\end{frame}


% %-------------------------------------------------------------------------------
% \section{Back up}
% %-------------------------------------------------------------------------------
% \appendix
%
% \begin{frame}[t]{}
%   \centering
%   \textcolor{tugreen}{\Huge{Back Up}}
% \end{frame}

\end{document}
